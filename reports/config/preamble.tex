\usepackage{textcase}
\newcommand{\var}[1]{\texttt{#1}}
\usepackage{float}
\usepackage{caption}
\usepackage{subcaption}
\usepackage{multicol} % {multicols}
\usepackage{fancyvrb}
\usepackage{fvextra}
  \fvset{
    fontsize=\footnotesize,
    frame=leftline,
    rulecolor=\color{gray},
    tabsize=4,
    breaklines=true,
  breakanywhere=true}
  \newcommand{\code}[1]{
    \subsection*{#1}
    \VerbatimInput{../L2/#1}
  }
\usepackage{fontspec}
\newcommand{\n}{\overline}

% \usepackage{array}
% Define typographic struts, as suggested by Claudio Beccari
%   in an article in TeX and TUG News, Vol. 2, 1993.
% \newcommand\Tstrut{\rule{0pt}{2.6ex}}         % = `top' strut
% \newcommand\Bstrut{\rule[-0.9ex]{0pt}{0pt}}   % = `bottom' strut
% src https://tex.stackexchange.com/questions/65127/extra-vertical-space-after-hline-causes-a-gap-in-the-right-border-of-an-array
  
% \usepackage{adjustbox}
%     \newcommand{\eqdef}{\xlongequal{\text{def}}}
%     \DeclareMathOperator{\tg}{tg}
%     \DeclareMathOperator{\arctg}{arctg}
%     \newcommand{\deleted}[1]{\overset{∘}{#1}}
%     \newcommand{\sequenceFrom}[3]{\left\{ #1 \right\}_{#2 = #3}^{∞}}
%     \newcommand{\sequence}[2]{\sequenceFrom{#1}{#2}{1}}
%     \newcommand{\sequenceN}[1]{\sequence{#1}{n}}
%     \newcommand{\sequenceNFrom}[2]{\sequenceFrom{#1}{n}{#2}}
% \newfontfamily\cyrillicfont[Script=Cyrillic]{Times New Roman}
%
%     \usepackage{pgfplots}
%     \pgfplotsset{compat=1.18}
%     \usepackage{mathrsfs}
%     \usetikzlibrary{arrows}
%     \usetikzlibrary{patterns.meta}
%     \usepackage{longtable, booktabs, array}
%     \usepackage{caption}
%     \usepackage{subcaption}
%     \AtBeginDocument{\renewcommand{\setminus}{\mathbin{\backslash}}}
%     \removenolimits{\int\iint}
  

% fancyvrb листинги

% https://tex.stackexchange.com/a/457403 - nolimits
% https://tex.stackexchange.com/a/140343 - setminus
% \usepackage[mdici]{mathdesign}
% \usepackage{upgreek}
% \renewcommand{\alpha}{\upalpha}
% \renewcommand{\beta}{\upbeta}
% \renewcommand{\delta}{\updelta}
% \renewcommand{\Delta}{\Updelta}
% \renewcommand{\lambda}{\uplambda}
% \renewcommand{\pi}{\uppi}
% \renewcommand{\varepsilon}{\upvarepsilon}
% \renewcommand{\varphi}{\upvarphi}
% \renewcommand{\psi}{\uppsi}
% \renewcommand{\xi}{\upxi}
\newfontfamily{\cyrillicfontsf}{XITS}
% \usepackage[mdici, greeklowercase=upright]{mathdesign}

% \newcommand{\veichTableThree}[9][Q]{\begin{tabular}{ccccc}
%   & \multicolumn{2}{c|}{$#1_2$} & \multicolumn{2}{c}{$\n #1_2$}   \Bstrut\\
%   $#1_3$    & #2  & \multicolumn{1}{|c|}{#3} & \multicolumn{1}{c|}{#4}  & #5  \Tstrut\Bstrut\\ \hline
%   $\n #1_3$ & #6  & \multicolumn{1}{|c|}{#7} & \multicolumn{1}{c|}{#8}  & #9  \Tstrut\\
%   & $\n #1_1$ & \multicolumn{2}{|c|}{$#1_1$} & $\n #1_1$        
%   \end{tabular}}
\usepackage{tikz}
\usepackage{algorithm}
\usepackage{algpseudocode}
\algrenewcommand\algorithmicrequire{\textbf{Вход:}}
\algrenewcommand\algorithmicensure{\textbf{Выход:}}
\newcommand{\Stop}{\textbf{stop}}
\newcommand{\Array}[2]{\textbf{array}[#1] \textbf{of} #2}
\newcommand{\Yield}{\textbf{yield }}
\newcommand{\Select}{\textbf{select }}
% \newcommand{\Return}{\textbf{return} }
\newcommand{\Real}{\textbf{real}}
\newcommand{\Top}{\textbf{top }}
\makeatletter
\renewcommand*{\ALG@name}{Алгоритм}
\makeatother


\DeclareMathOperator{\sgn}{sgn}
\DeclareMathOperator{\diag}{diag}

\makeatletter
\newcommand{\Scalecenter}[3]{#1{\mathpalette\Scalecenter@{{#2}{#3}}}}
\newcommand{\Scalecenter@}[2]{\Scalecenter@@#1#2}
\newcommand{\Scalecenter@@}[3]{%
  \vcenter{\hbox{\scalebox{#2}{$\m@th#1#3$}}}%
}
\newcommand{\striangle}{\Scalecenter{\mathbin}{0.5}{\triangle}}
\makeatother

% https://tex.stackexchange.com/questions/9497/start-new-page-with-each-section
\usepackage{titlesec}
\newcommand{\sectionbreak}{\clearpage}

\DeclareMathOperator{\func}{func}

% \newcommand{\va}{\var{a}}
% \newcommand{\vb}{\var{b}}
% \newcommand{\vc}{\var{c}}
% \newcommand{\vd}{\var{d}}
% \newcommand{\ve}{\var{e}}
% \newcommand{\vf}{\var{f}}
% \newcommand{\vg}{\var{g}}
% \newcommand{\vh}{\var{h}}
% \newcommand{\vi}{\var{i}}
% \newcommand{\vj}{\var{j}}
% \newcommand{\vk}{\var{k}}

% \usepackage{forest}
% \usetikzlibrary{arrows.meta}
% \tikzset{
%   0 my edge/.style={densely dashed, my edge},
%   my edge/.style={-{Stealth[]}},
% }
% \forestset{
%   BDT/.style={
%     for tree={
%       if n children=0{}{circle},% use a circle unless there are 0 children
%       draw,% draw every node
%       edge={
%         my edge,% use the my edge style for edges (with the arrow)
%       },
%       if n=1{
%         edge+={0 my edge},% if the child is the first one, add the 0 my edge style (dashed)
%       }{},
%       font=\sffamily,% use sans serif for node text
%     }
%   },
% }
% \forestset{
%   SYNT/.style={
%     for tree={
%       if n children=0{}{circle},% use a circle unless there are 0 children
%       draw,% draw every node
%       edge={
%         my edge,% use the my edge style for edges (with the arrow)
%       },
%       font=\sffamily,% use sans serif for node text
%     }
%   },
% }
% \DeclareMathOperator{\pdnf}{\text{СДНФ}}
% \DeclareMathOperator{\pcnf}{\text{СКНФ}}

\AtBeginDocument{\renewcommand{\div}{\operatorname{div}}}